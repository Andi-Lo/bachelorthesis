
%Eidesstattliche Erklärung
\begin{newpage}
	\hline
	\vspace*{\fill}
	\section*{Eidesstattliche Erklärung}
	Hiermit versichere ich, die vorliegende Arbeit selbstständig und unter ausschließlicher Verwendung der angegebenen Literatur und Hilfsmittel erstellt zu haben. Die Arbeit wurde bisher, in gleicher oder ähnlicher Form, keiner anderen Prüfungsbehörde vorgelegt und auch nicht veröffentlicht.\\

	\vspace{3cm}
	\begin{tabular*}{\textwidth}{c@{\extracolsep\fill}cc}
	\cline{1-1}
	\cline{3-3}
	\\
	\ \ \ \ \ \ \ \ \ Unterschrift\ \ \ \ \ \ \ \ \ \ & & \ \ \ \ \ \ \ \ \ Ort, Datum\ \ \ \ \ \ \ \ \ \\
	\end{tabular*}
\end{newpage}


%Vorwort, Zusammenfassung(Abstract), Danksagung(Acknowledgements)
\begin{newpage}
	\hline
	\vspace*{\fill}
	\section*{Vorwort}
	\subsection*{Zusammenfassung}
		Zeit wird als ein kostbares und begrenztes Gut empfunden, weswegen wir in sämtlichen Lebensbereichen auf deren Ersparnis und Nutzungsoptimierung bedacht sind. Dies gilt auch für aktivitäten im modernen Web. Allerdings laden auch im Jahr 2015 Webanwendungen und Webseiten immer noch sehr langsam. Diese Arbeit befasst sich mit den Ursachen und Gründen für diese Problematik und soll Lösungsstrategien aufzeigen, die dem Smartphone Anwender eine schnellere Online Erfahrung ermöglichen. Dafür werden Grundlagen erklärt, \texttt{Best Practices} erläutert und beschrieben, wie der Weg zur performanten Webanwendung aussehen kann.\\
		Diese Methodiken zur Verbesserung der Performance wurden im Rahmen eines eigenen Projekts umgesetzt und erprobt. Dabei wurde ein automatisiertes Testverfahren verwendet und Daten wie Load Time, Speed Index, Render Start, Time to Visually Complete und andere Metriken über den Projektzeitlauf erfasst. Bei der Datenauswertung ergab sich eine signifikante Verringerung dieser Werte und damit verbunden eine drastische Reduzierung der Seitenladezeit von 7,6 auf unter 1,5 Sekunden.\\
		Die hier beschriebenen Methodiken erlauben eine umfassende Steigerung der Nutzererfahrung für Webseiten im Internet. Die Arbeit schließt mit der Erkenntnis, dass die Voraussetzung für Web Performance weniger die Programmierung ist, sondern in erster Linie eine Veränderung der Unternehmenskultur stattfinden muss, um Web Performance richtig umsetzen zu können.

	\subsection*{Danksagung}
		Mein besonderer Dank für das inhaltliche Überprüfen und die Korrektur der Arbeit geht an:
    \begin{flushleft}\large
		  \hspace*{0.5cm} Christina Negele\\
		  \hspace*{0.5cm} Michael Lorer
    \end{flushleft}
\end{newpage}


\newpage