\section{Einleitung} % (fold)
\label{sec:einleitung}

% section einleitung (end)



\section{Motivation} % (fold)
\label{sub:motivation}

	Larry Page, CEO und Mitgründer von Google, sagt:
	\begin{quote}
		\textit{"`As a product manager you should know that speed is the number one feature."'}\autocite{}
	\end{quote}
	Niemand mag es zu warten, auch nicht auf eine Website. Die Studie "`The Psychology of Web Performance"' kam bereits im Jahr 2008 schon auf folgende Ergebnisse:

	\begin{quote}\itshape
		"`Slow web pages lower perceived credibility and quality. Keep your page load times below tolerable attention thresholds, and users will experience less frustration, lower blood pressure, deeper flow states, higher conversion rates, and lower bailout rates. Faster websites are actually perceived to be more interesting and attractive."' \autocite{webOpti08}
	\end{quote}

	Tatsächlich ist für Google Geschwindigkeit alles. Das haupt Vermarktungsargument für den Chrome Browser war damals: er sei schneller als die Konkurenz.	Deshalb hat Google im Jahr 2010 angekündigt, dass Geschwindigkeit in die Berechnung des \texttt{Page Rankings} mit einfließt.

	\begin{quote}\itshape
		"`Faster sites create happy users and we've seen in our internal studies that when a site responds slowly, visitors spend less time there. [...] Recent data shows that improving site speed also reduces operating costs. Like us, our users place a lot of value in speed — that's why we've decided to take site speed into account in our search rankings"'\autocite{google10}
	\end{quote}

	Aktuell (2015) geht Google sogar noch einen Schritt weiter und informiert tausende Webmaster per E-Mail über die schlechte Usability ihrer Websites für mobile Besucher und warnt ausdrücklich vor dementsprechend "`angepassten Rankings"'.\autocite{t3n15}
	Im Hinblick auf die Zukunft wird der Marktanteil an mobilen Internetnutzern noch weiter wachsen und die Optimierung der Ladezeiten gewinnt dadurch noch mehr an bedeutung. Zwischen 2011 und 2014 stieg die Anzahl der Smartphone nutzer von 18\% auf 50\% an. Dies ist ein Wachstum von 32\% innerhalb von nur 3 Jahren.\autocite{tns14}\\

	\begin{figure}[htbp]
		\begin{center}
			\includegraphics[width=0.75\textwidth]{smartphoneUsage.png}
		\end{center}
		\caption{Gerätenutzung in der Gesamtbevölkerung (2011 – 2014)\autocite{tns14}}
		\label{fig:geraetenutzung}
	\end{figure}

	Die Antwort auf diesen Trend läutete eine Ära ein, die wir heute unter dem Namen \texttt{Responsive Webdesign} kennen. "`Responsive"' muss aber sehr viel mehr bedeuten, als nur eine angepasste Darstellung für eine bestimmte Art von Gerät. "`Two out of three mobile shoppers expect pages to load in 4 seconds or less."' \autocite{radware13}. Der Anwender erwartet also auf dem Smartphone änliche oder gleiche Ladezeiten wie er auch von der Nutzung eines Desktop-Pc's gewohnt ist.

	Der Inhalt einer Seite muss darum so aufbereitet werden, dass dieser auch auf Geräten mit langsamer Internetverbindung, hoher Latenz und einem begrenzten Datentarif, in einer für den Anwender annehmbaren Geschwindigkeit, angezeigt werden kann.\\

% subsection motivation (end)


\subsection{Zielsetzung} % (fold)
\label{sub:zielsetzung}
	Um gängige Methoden und Techniken der Ladezeit Optimierung anzuwenden wird das Projekt anhand der Website \url{http://andreaslorer.de} durchgeführt. Das Ziel ist es, die Ladezeit der Website auf dem Smartphone, als auch auf dem Desktop von 10 Sekunden auf unter 1 Sekunden zu verringern. Mit Ladezeit ist dabei nicht die Zeit gemeint, die benötigt wird um die Website komplett zu laden, sondern die Zeit bis ein erste visuelle Rückmeldung für den Anwender zu sehen ist. Diese vom Anwender wahrgenommene Rückmeldung nennt man auch "`Perceived Performance"' und bedeutet, dass die Ladezeit als schneller empfunden wird, als es eigentlich laut Messwerten der Fall ist. Näheres dazu wird in dem Kapitel \ref{..fehlt noch} beschrieben.\\
	% todo: referencing chapter
	Es soll herausgefunden werden, was die "`Best Practices"' sind um die Ladezeit zu minimieren, wie ein moderner "`workflow"' aussehen kann, damit eine Webanwendung schon bei seiner Entstehung schnell ladet und im Projektverlauf schnell bleibt. Des weiteren soll erklärt werden, was für herausforderungen es zu meistern gilt um eine schnelle Webanwendung zu erreichen, welche Tools es gibt und welche Vor- oder Nachteile diese mit sich bringen.\\
	Diese Arbeit befasst sich nicht, mit der Geschwindigkeit von Datenbanken, SQL-Abfragen oder sonstigen problemen die durch einen Engpass ein schnelles Laden der Seite verhindern könnten.

% subsection zielsetzung (end)

\section{Eigene Leistung} % (fold)
\label{sub:eigene_leistung}
	Meine Leistung besteht darin, einen Leitfaden zu erstellen, der einen Gesamtüberblick ermöglicht. Die Arbeit soll es dem Leser ermöglichen Fehler in der Struktur von Webanwendungen zu finden, die für die Geschwindigkeit hinderlich sind.
	% todo: überarbieten, nicht greifbar was meine Leistung ist
	% Folgender Ansatz für die Argumentation könnte aufgebracht werden:
	% Meine Leistung besteht darin, aufzuzeigen wie ich die Webseite http://andreaslorer.de von einer Ladezeit von rund 10 sekunden auf unter 1 Sekunden bringen könnte, welche Schwirigkeiten es zu meistern gilt und so weiter

% subsection eigene_leistung (end)


\section{Ist-Zustand} % (fold)
\label{sec:Ist-Zustand}
	Die Webseite ist auf einem \texttt{shared Hosting}
	\footnote{Bei shared Hosting werden mehrere Websites von verschiedenen Website-Betreibern von dem gleichen Webserver gehostet. Bei Shared Hosting teilen sich in der Regel Hunderte andere Websites einen Server \autocite{itWissen}} 
	aufgesetzt und antwortet auf ein Ping Kommando in rund 13ms. Dadurch, dass es keine Möglichkeit gibt \texttt{root Rechte} \footnote{Standardmäßig existiert unter Linux immer ein Konto für den Benutzer "`root"' mit der User-ID 0. Dies ist ein Systemaccount mit vollem Zugriff auf das gesamte System, und damit auch auf alle Dateien und Einstellungen aller Benutzer \autocite{ubuntu14}} auf einem shared hosting zu bekommen können so manche serverseitige Einstellungen nicht durchgeführt werden. Diese werden dann zwar Aufgezeigt, aber kommen für dieses Projekt nicht zum Einsatz.\\
	Die Website hat als Ausgangsbasis einen gängigen Aufbau. Sie besteht aus einer Bilder Gallerie basierend auf PHP und dem Bootstrap Framework.
% section  (end)
