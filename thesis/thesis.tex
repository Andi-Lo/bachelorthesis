% Bachelor Thesis Webperformance für den mobilen Endanwender
% von Andreas Lorer
% 2015
% Studiengang: Angewandte Informatik - HS Ravensburg-Weingarten
\documentclass[a4paper,11pt,singlespacing]{article}
\usepackage[left=2.5cm,right=2.5cm,top=2.5cm]{geometry}
\usepackage[hyphens]{url}
\usepackage[backend=biber,style=alphabetic,backref=true,citecounter=true,citestyle=authoryear]{biblatex}
\usepackage{csquotes}
\usepackage{hyperref}
\usepackage{setspace}
\usepackage[utf8]{inputenc}
\usepackage[T1]{fontenc}
\usepackage[ngerman]{babel}
\usepackage{color}
\usepackage{hyperref}
\usepackage{pdflscape}
\usepackage{graphicx}
\usepackage{listings}
\usepackage{csquotes}
\addbibresource{sections/zitate.bib}
\pagestyle{myheadings}
\markright{\centerline{Webperformance für den mobilen Endanwender}}
\author{Andreas Lorer}

\graphicspath{{images/}}            
\setlength{\parindent}{0ex}  	% Absatzeinrueckung verhindern

\definecolor{mygreen}{rgb}{0,0.6,0}
\definecolor{mygray}{rgb}{0.5,0.5,0.5}
\definecolor{mymauve}{rgb}{0.58,0,0.82}

\newenvironment{italicquotes}
{\begin{quote}\itshape}
{\end{quote}}

\begin{document}
  % document styles for listings
  \lstset{
    backgroundcolor=\color{white},
    basicstyle=\footnotesize\ttfamily,
    breakatwhitespace=true,
    breaklines=true,
    numbers=left,
    numbersep=5pt,
    deletekeywords={event}, 
    numberstyle=\tiny,
    showspaces=false,
    showtabs=false,
    language=html,
    keywordstyle=\bfseries\color{mymauve},
    commentstyle=\itshape\color{mygreen},
    columns=fullflexible,
    xleftmargin=0.5cm,
    xrightmargin=1cm,
    frame=lr,
    framesep=8pt,
    framerule=0pt
  }

	%Deckblatt
	\newcommand{\headline}{Bachelor-Thesis}
	\newcommand{\subheadline}{Webperformance für den mobilen Endanwender}
	\begin{titlepage}
  \sffamily
  \setlength{\tabcolsep}{0mm}
  \begin{tabular*}{\textwidth}{l@{\extracolsep\fill}r} 

  %\hspace{-0.4cm}
  %\includegraphics[width=6cm]{Bilder/logo_welle_en}  % Englische Version des Logos 
  \includegraphics[width=5cm]{../deckblatt/images/hs-wgt-de.png} % Deutsche Version des Logos

    &
  \raisebox{3mm}{
  \begin{tabular}{r}
    \rule{0cm}{0.5cm}
    Studiengang Angewandte Informatik\\[0.5mm]
    Fakultät Elektrotechnik und Informatik \\
  \end{tabular}}
  \end{tabular*}
  \setlength{\tabcolsep}{6pt}

  \vspace*{2cm}
  \begin{center}
      \textbf{\Large{Bachelor-Thesis}}\\[1cm]
    \begin{doublespace}
      \textbf{\LARGE{Web Performance für den mobilen Endanwender}}\\[1cm]
    \end{doublespace}
    \includegraphics[width=0.5\textwidth]{../deckblatt/images/webperf_logo.jpg}\\[0.5cm]
    \vspace*{1cm}
    % \vspace*{2cm}
    \large{zur Erlangung des akademischen Grades}\\[2mm]
    \large{Bachelor of Science}\\
  \end{center}

  %\vfill
  \vspace{0.5cm}
  \begin{center}

  vorgelegt von:\\[5mm]
  {\Large Andreas Lorer} \\[5mm]
    Wilhelmstraße 4\\
    88250 Weingarten\\
    \today \\[2cm]
  {\normalsize
    \begin{tabular}{rl}
    1. Gutachter:   & Prof. Klemens Ehret \\
    2. Gutachter:   & Lukas Michna\\
    \end{tabular}
  }
  \end{center}
  \vfill
\end{titlepage}

	
	%Inhaltsverzeichnis
	\tableofcontents 
	\thispagestyle{empty}
	\pagebreak

    \section{Einleitung} % (fold)
\label{sec:einleitung}

% section einleitung (end)



\section{Motivation} % (fold)
\label{sub:motivation}

	Larry Page, CEO und Mitgründer von Google, sagt:
	\begin{quote}
		\textit{"`As a product manager you should know that speed is the number one feature."'}\autocite{}
	\end{quote}
	Niemand mag es zu warten, auch nicht auf eine Website. Die Studie "`The Psychology of Web Performance"' kam bereits im Jahr 2008 schon auf folgende Ergebnisse:

	\begin{quote}\itshape
		"`Slow web pages lower perceived credibility and quality. Keep your page load times below tolerable attention thresholds, and users will experience less frustration, lower blood pressure, deeper flow states, higher conversion rates, and lower bailout rates. Faster websites are actually perceived to be more interesting and attractive."' \autocite{webOpti08}
	\end{quote}

	Tatsächlich ist für Google Geschwindigkeit alles. Das haupt Vermarktungsargument für den Chrome Browser war damals: er sei schneller als die Konkurenz.	Deshalb hat Google im Jahr 2010 angekündigt, dass Geschwindigkeit in die Berechnung des \texttt{Page Rankings} mit einfließt.

	\begin{quote}\itshape
		"`Faster sites create happy users and we've seen in our internal studies that when a site responds slowly, visitors spend less time there. [...] Recent data shows that improving site speed also reduces operating costs. Like us, our users place a lot of value in speed — that's why we've decided to take site speed into account in our search rankings"'\autocite{google10}
	\end{quote}

	Aktuell (2015) geht Google sogar noch einen Schritt weiter und informiert tausende Webmaster per E-Mail über die schlechte Usability ihrer Websites für mobile Besucher und warnt ausdrücklich vor dementsprechend "`angepassten Rankings"'.\autocite{t3n15}
	Im Hinblick auf die Zukunft wird der Marktanteil an mobilen Internetnutzern noch weiter wachsen und die Optimierung der Ladezeiten gewinnt dadurch noch mehr an bedeutung. Zwischen 2011 und 2014 stieg die Anzahl der Smartphone nutzer von 18\% auf 50\% an. Dies ist ein Wachstum von 32\% innerhalb von nur 3 Jahren.\autocite{tns14}\\

	\begin{figure}[htbp]
		\begin{center}
			\includegraphics[width=0.75\textwidth]{smartphoneUsage.png}
		\end{center}
		\caption{Gerätenutzung in der Gesamtbevölkerung (2011 – 2014)\autocite{tns14}}
		\label{fig:geraetenutzung}
	\end{figure}

	Die Antwort auf diesen Trend läutete eine Ära ein, die wir heute unter dem Namen \texttt{Responsive Webdesign} kennen. "`Responsive"' muss aber sehr viel mehr bedeuten, als nur eine angepasste Darstellung für eine bestimmte Art von Gerät. "`Two out of three mobile shoppers expect pages to load in 4 seconds or less."' \autocite{radware13}. Der Anwender erwartet also auf dem Smartphone änliche oder gleiche Ladezeiten wie er auch von der Nutzung eines Desktop-Pc's gewohnt ist.

	Der Inhalt einer Seite muss darum so aufbereitet werden, dass dieser auch auf Geräten mit langsamer Internetverbindung, hoher Latenz und einem begrenzten Datentarif, in einer für den Anwender annehmbaren Geschwindigkeit, angezeigt werden kann.\\

% subsection motivation (end)


\subsection{Zielsetzung} % (fold)
\label{sub:zielsetzung}
	Um gängige Methoden und Techniken der Ladezeit Optimierung anzuwenden wird das Projekt anhand der Website \url{http://andreaslorer.de} durchgeführt. Das Ziel ist es, die Ladezeit der Website auf dem Smartphone, als auch auf dem Desktop von 10 Sekunden auf unter 1 Sekunden zu verringern. Mit Ladezeit ist dabei nicht die Zeit gemeint, die benötigt wird um die Website komplett zu laden, sondern die Zeit bis ein erste visuelle Rückmeldung für den Anwender zu sehen ist. Diese vom Anwender wahrgenommene Rückmeldung nennt man auch "`Perceived Performance"' und bedeutet, dass die Ladezeit als schneller empfunden wird, als es eigentlich laut Messwerten der Fall ist. Näheres dazu wird in dem Kapitel \ref{..fehlt noch} beschrieben.\\
	% todo: referencing chapter
	Es soll herausgefunden werden, was die "`Best Practices"' sind um die Ladezeit zu minimieren, wie ein moderner "`workflow"' aussehen kann, damit eine Webanwendung schon bei seiner Entstehung schnell ladet und im Projektverlauf schnell bleibt. Des weiteren soll erklärt werden, was für herausforderungen es zu meistern gilt um eine schnelle Webanwendung zu erreichen, welche Tools es gibt und welche Vor- oder Nachteile diese mit sich bringen.\\
	Diese Arbeit befasst sich nicht, mit der Geschwindigkeit von Datenbanken, SQL-Abfragen oder sonstigen problemen die durch einen Engpass ein schnelles Laden der Seite verhindern könnten.

% subsection zielsetzung (end)

\section{Eigene Leistung} % (fold)
\label{sub:eigene_leistung}
	Meine Leistung besteht darin, einen Leitfaden zu erstellen, der einen Gesamtüberblick ermöglicht. Die Arbeit soll es dem Leser ermöglichen Fehler in der Struktur von Webanwendungen zu finden, die für die Geschwindigkeit hinderlich sind.
	% todo: überarbieten, nicht greifbar was meine Leistung ist
	% Folgender Ansatz für die Argumentation könnte aufgebracht werden:
	% Meine Leistung besteht darin, aufzuzeigen wie ich die Webseite http://andreaslorer.de von einer Ladezeit von rund 10 sekunden auf unter 1 Sekunden bringen könnte, welche Schwirigkeiten es zu meistern gilt und so weiter

% subsection eigene_leistung (end)


\section{Ist-Zustand} % (fold)
\label{sec:Ist-Zustand}
	Die Webseite ist auf einem \texttt{shared Hosting}
	\footnote{Bei shared Hosting werden mehrere Websites von verschiedenen Website-Betreibern von dem gleichen Webserver gehostet. Bei Shared Hosting teilen sich in der Regel Hunderte andere Websites einen Server \autocite{itWissen}} 
	aufgesetzt und antwortet auf ein Ping Kommando in rund 13ms. Dadurch, dass es keine Möglichkeit gibt \texttt{root Rechte} \footnote{Standardmäßig existiert unter Linux immer ein Konto für den Benutzer "`root"' mit der User-ID 0. Dies ist ein Systemaccount mit vollem Zugriff auf das gesamte System, und damit auch auf alle Dateien und Einstellungen aller Benutzer \autocite{ubuntu14}} auf einem shared hosting zu bekommen können so manche serverseitige Einstellungen nicht durchgeführt werden. Diese werden dann zwar Aufgezeigt, aber kommen für dieses Projekt nicht zum Einsatz.\\
	Die Website hat als Ausgangsbasis einen gängigen Aufbau. Sie besteht aus einer Bilder Gallerie basierend auf PHP und dem Bootstrap Framework.
% section  (end)

	
\section{Begriffe}
\label{sec:begriffe}
	\subsection{Pattern und Antipattern} % (fold)
	\label{sub:pattern_und_anti_pattern}
		"`Entwurfsmuster (englisch design patterns) sind bewährte Lösungsschablonen für wiederkehrende Entwurfsprobleme sowohl in der Architektur als auch in der Softwarearchitektur und -entwicklung. Sie stellen damit eine wiederverwendbare Vorlage zur Problemlösung dar, die in einem bestimmten Zusammenhang einsetzbar ist."'\autocite{pattern15}
		\\

		"`Während "`Design Patterns"' in der Software-Entwicklung allgemein übliche und bekannte Ansätze sind, um Probleme zu lösen, sind Anti-Patterns Negativ-Beispiele – die zeigen, wie man es nicht macht - von bereits durchgeführten, gescheiterten Projekten, die dem erkennenden Mitarbeiter zielgerichtete Hinweise darauf geben, wie die Aufgabenstellung besser gelöst werden könnte. Als Synonym ist auch der Begriff Negativmuster im Gebrauch. Es ist tatsächlich möglich, daß das, was gestern noch als allgemein gangbarer Lösungsweg bezeichnet wurde, heute schon ein "`Antipattern"' ist [...]"' \autocite{Stepken06}

	% subsubsection pattern_und_anti_pattern (end)


	\subsection{Latenz} % (fold)
	\label{sub:latenz}
		Latenz bezeichnet die Verzögerung, bis ein Paket von Sender A zu Empfänger B gelangt ist.
	% subsection latenz (end)


	\subsection{Round Trip Time (RTT)} % (fold)
	\label{sub:round_trip_time_}
		"`Round Trip Time"' wird im Deutschen Paketumlaufzeit genannt. Es bezeichnet die Zeit die ein Datenpaket braucht um in einem Netzwerk von Sender A zu Empfänger B und wieder zurück zu gelangen. Bei einer Latenz von 100ms würde die RTT folglich 200ms betragen (Annahme: Hin- und Rückweg haben die selbe Zeit).
	
	% subsection round_trip_time_ (end)


	\subsection{Http/1.1}
	\label{sub:http_1_1}
		Der für diese Arbeit wichtige Aspekt der HTTP/1.1 Spezifikation ist die Limitierung von Verbindungen pro Domain Name. Dabei weicht die Limitierung zwischen den Browsern ab und reicht von 6 (Google Chrome) bis 13 (Internet Explorer 11) parallelen TCP Verbindungen. Eine Übersichtsliste ist hier zu finden \url{http://www.browserscope.org/?category=network}
	% subsection http_1_1 (end)


	\subsection{TCP Three Way Handshake}
	\label{sub:tcp_three_way_handshake}
		TCP ist das meistgenutzte Verbindungsprotokoll im Internet. Auf diesem Protokoll wird der HTTP Request aufgebaut, der die eigentlichen Daten enthält.
		Bevor Daten zwischen Server und Browser ausgetauscht werden können, muss eine Verbindung aufgebaut werden. Abbildung ~\ref{fig:three_way_handshake} beschreibt den Prozess des Verbindungsaufbaus.\footnote{Für ein tieferes Verständnis empfiehlt sich dieser Artikel: \href{http://chimera.labs.oreilly.com/books/1230000000545/ch02.html}{High Performance Browser Networking - Chapter 2: Building Blocks of TCP}} 

		\begin{figure}[htbp]
			\begin{center}
				\includegraphics[width=0.6\textwidth]{three_way_handshake.jpg}
				\caption{Three-Way-Handshake zum Aufbau einer TCP Verbindung zwischen Browser und Server (Eigene Abbildung nach: \autocite{bos})}
				\label{fig:three_way_handshake}
			\end{center}
		\end{figure}
	% subsection tcp_three_way_handshake (end)

	\subsection{TCP Slow Start} % (fold)
	\label{sub:tcp_slow_start}

		Ein Round Trip kann nicht beliebig viele Bytes transportieren sondern ist durch die sogenannte "`Congestion Window Size"'\footnote{engl. congestion: Stauung, Überlastung, Anhäufung} limitiert. Der Überbegriff für dieses Verhalten nennt sich "`Slow Start"'

		\begin{figure}[htbp]
			\begin{center}
				\includegraphics[width=\textwidth]{congestion_window_size.jpg}
				\caption{Congestion Control und Congestion Avoidance (Eigene Abbildung nach \autocite{grigorikSlowStart})}
				\label{fig:congestion_window_size}
			\end{center}
		\end{figure} 

		\begin{itemize}
			\item Congestion Control: Nach dem eine neue Verbindung per TCP aufgebaut wurde, können weder Server noch Client wissen, wie schnell die Verfügbare Bandbreite ist, mit der Daten ausgetauscht werden können. Um das Netzwerk, vor einem Datenstau zu schützen, wird mit einem sehr niedrigen Wert begonnen, der dann ansteigt bis das Limit erreicht ist. Dieses Verhalten nennt sich auch "`Congestion Control"' und verhindert das Aufstauen von Daten.
			\item Congestion Window Size: Diese Größe bestimmt, wieviel Bytes der pro Segmente geschickt werden darf, bis diese vom Empfänger per \texttt{ACK} (acknowledgement) bestätigt werden müssen. Die Größe der Segmente ist Standardmäßig 1460 bytes und die Rate bis zum ACK ist im April 2013 von 4 auf 10 Segmente erhöht worden.\autocite{grigorikSlowStart}. In der Grafik wird davon ausgegangen, das der erste Round Trip 4 Segmente senden darf. Die Datenrate Wächst exponentiell an, damit möglichst schnell die volle Bandbreite nutzbar ist.\\
			\item Congestion Avoidance bedeutet, dass sich die Datenrate wieder um ein Vielfaches verringert, falls es zu einem Paketverlust kommt. Da es besonders bei WLAN oder Mobilfunknetzen des öfteren zu Packetverlusten kommen kann ist dieser Aspekt besonders hervorzuheben, denn er verzögert das erreichen der maximal möglichen Datenrate.
		\end{itemize}

		Slow Start bedeutet also aus sicht der Performance, dass bei einer neuen TCP Verbindung nicht die maximale Bandbreite zu Verfügung steht. Bei größeren Dateien wird zwar durch das exponentielle Wachstum das Maximum schnell erreicht, gerade aber bei kleineren Dateien mit wenigen Kilobyte ist dies oft nicht der Fall.

	% subsubsection tcp_slow_start (end)


	\subsection{Content Delivery Network (CDN)} % (fold)
	\label{sub:content_delivery_network}
		Ein Content Delivery Network (CDN), oder auch Content Distribution Network genannt, ist ein Netz lokal verteilter und über das Internet verbundener Server, mit dem Inhalte ausgeliefert werden. CDN-Knoten sind auf viele Orte verteilt um Anfragen (Requests) von End-Nutzern nach Inhalten (Content) möglichst ökonomisch zu bedienen. Große CDNs unterhalten tausende Knoten mit zehntausenden Servern.\autocite{wikipediaCDN}

		\begin{figure}[htbp]
			\begin{center}
				\includegraphics[width=0.6\textwidth]{cdn_network.jpg}
				\caption{Schematische Darstellung eines CDN (Eigene Abbildung nach \autocite{ritz14})}
				\label{fig:cdn_network}
			\end{center}
		\end{figure}
		
	% subsection content_delivery_network (end)


	\subsection{Above The Fold} % (fold)
	\label{sub:above_the_fold}
		Damit ist der auf einem Bildschirm sichtbare Bereich vor dem Scrollen gemeint. Diesem Bereich wird eine besondere Wichtigkeit zugesprochen.
		\begin{figure}[htbp]
			\begin{center}
				\includegraphics[width=0.3\textwidth]{above_the_fold.jpg}
				\caption{Darstellung des sichtbaren Bereichs vor dem Scrollen}
				\label{fig:above_the_fold}
			\end{center}
		\end{figure}

		\begin{quote}
			 \textit{"`In an analysis of 57,453 eyetracking fixations, we found that there was a dramatic drop-off in user attention at the position of the page fold. Elements above the fold were seen more than elements below the fold: the 100 pixels just above the fold were viewed 102\% more than the 100 pixels just below the fold."'} \autocite{nng15}
		\end{quote}

		Wichtige Informationen oder Navigationselemente sind meistens dort zu finden. Eine Webseite die nach dem Paradigma des Responsive-Webdesign aufgebaut ist kann dabei 3 oder mehrere Ansichten haben die alle einen unterschiedlichen "`above the fold"' bereich haben. Eine Anwendung kann aber auch unterschiedliche Seiten haben, auf dem der Anwender beim Aufrufen der Seite landen kann. Zum Beispiel wenn dieser an- oder abgemeldet ist. Paradebeispiel dafür sind Facebook oder Twitter.

	% subsubsection above_the_fold (end)



	\subsection{Perceived Performance} % (fold)
	\label{sub:perceived_performance}
		\begin{figure}[htbp]
			\begin{center}
				\includegraphics[width=0.9\textwidth]{perceived_performance.jpg}
				\caption{Zwei Seiten im Vergleich (Eigene Abbildung via webpagtest.org)}
				\label{fig:perceived_performance}
			\end{center}
		\end{figure}

		Abbildung \ref{fig:perceived_performance} zeigt die Seiten A und B, mit nahezu identischer Ladezeit. Der Unterschied besteht darin, dass Seite B bereits nach 1.5 Sekunden eine erste visuelles Rückmeldung für den Anwender zu sehen ist wohingegen Seite A erst nach 5.5 Sekunden dem Anwender zeigt, dass sie überhaupt ladet.
		"`Perceived Performance"' steht also für die Zeit bis ein erste visuelle Rückmeldung für den Anwender zu sehen ist und bedeutet, dass die Ladezeit als schneller empfunden wird, als es eigentlich laut Messwerten der Fall ist. Warum diese "`Perceived Performance"' für eine Webanwendung so wichtig ist zeigen mehrere Studien, deren Daten in folgender Infographik aufbereitet sind.

		\begin{figure}[htbp]
			\begin{center}
				\includegraphics[width=0.7\textwidth]{effect_of_slow_loadtimes.jpg}
				\caption{Einfluss und Effekt einer langsamen Seite auf den Anwender (Eigene Abbildung nach Daten von: \autocite[p. 8]{radware14})}
				\label{fig:effect_of_slow_loadtimes}
			\end{center}
		\end{figure}

		Bereits kleine Verbesser- oder Verschlechterungen der Ladezeit können einen großen Einfluss in auf den Anwender haben. Yahoo hat herausgefunden, dass wenn eine Seite um nur 400 Millisikunden schneller ist, sich der Traffic um 9\% erhöhte.\autocite{stefanov08} 57\% der Online Konsumenten haben eine Seite, die länger als 3 Sekunden ladet bereits wieder verlassen. 78\% der Anwender empfinden sogar Zorn oder Stress wenn eine Seite nicht Ladet oder dies nicht ersichtlich ist.

	% subsubsection perceived_performance (end)
	\pagebreak
% subsection begriffe (end)



\section{Die 1000 ms Barriere} % (fold)
\label{sec:die_1000_ms_barriere}
	Das Ziel dieser Arbeit, die 1000 Millisekunden Barriere zu durchbrechen, wurde nicht durch einen Zufall gewählt. Der Anwender nimmt die Geschwindigkeit einer Seite subjektiv wahr. Sie wird in der folgenden Grafik interpretiert:

	\begin{figure}[htbp]
		\begin{center}
			\includegraphics[width=\textwidth]{human_perception.jpg}
			\caption{Zeit und Wahrnehmung durch den Anwender (Eigene Abbildung nach Daten von: \autocite{grigorikHumanPerception})}
			\label{fig:human_perception}
		\end{center}
	\end{figure}

	Wie zu sehen ist bleiben gerade einmal eine Sekunde, bevor das Gehirn uns sagt, man solle doch einer anderen Aufgabe nachgehen bis der Ladevorgang abgeschlossen ist. Der Anwender verlangt visuelle Rückmeldung um "`am Ball zu bleiben"', dies wurde bereits in Punkt ~\ref{sub:perceived_performance} "`Perceived Performance"' angesprochen. Auf vielen Webseiten sieht man deshalb, Ladebalken oder sogenannte \texttt{Spinner}, die dem Anwender sagen, dass der Ladevorgang in gange, aber noch nicht abgeschlossen ist.

	Um das Ziel von einer Sekunde Ladezeit bis zum ersten Render zu erreichen, ist es nötig zu verstehen, womit die meiste Zeit beim Aufrufen einer Webanwendung verbracht wird. Bevor eine Seite mittels Smartphone vom Browser dargestellt wird läuft eine ganze Reihe von Prozessen ab.



	\subsection{Touch Event} % (fold)
	\label{sub:touch_event}
		Der Aufruf einer Seite über das Smartphone erfolgt über ein Touch Event auf einen Link, Button oder die Seite wird per URL aufgerufen. Hierbei können je nach Gerät zwischen 50 (IPhone 5) und 123 Millisekunden (Moto X - Android) zwischen der Berührung des Touch Screen und dem Registrieren des Events vergehen.\autocite{venturebeat} Der Browser wartet allerdings nochmals bis zu 300 Millisekunden, denn er muss abwarten ob vielleicht noch ein zweiter Finger aufgelegt wird (Multitouch), oder ob der Anwender Scrollen oder Zoomen möchte.\autocite{google11}

		Dieses Verhalten lässt sich bei vielen Browsern per \texttt{Meta Tag} abstellen:

		\begin{lstlisting}
			<meta name="viewport" content="user-scalable=no">
		\end{lstlisting}

		Dies setzt natürlich voraus, dass die Webanwendung kein Zoomen benötigt um sie zu bedienen! Gerade bei älteren Webseiten trifft das oftmals nicht zu, da sie keine für das Smartphone angepasste Ansicht haben (responsive view). Eine vollständige Liste mit Meta Tags für die verschiedenen Browser ist der Fußnote zu entnehmen.\footnote{Suppressing 300ms delay for touchscreen interactions: \url{http://tinyurl.com/psj5nxz}}

	% subsection touch_event (end)

	\subsection{Netzwerke} % (fold)
	\label{sub:netzwerke}
		Warum gerade das nutzen des Internets per Smartphone so langsam sein kann (und oftmals ist) liegt zu einem Großteil am Netzwerk. Eine Studie untersuchte die Top eine Millionen Webseiten des Internets auf ihre Ladezeiten. Dabei wurde eine Verbindung von 5 Mbit/s und 28ms RTT benutzt. Eine RTT von 28 ms ist sehr schnell, vergleicht man sie zum Beispiel mit der Latenz des 3G Netzes (Abbildung ~\ref{fig:mobile_networks}). Diese Studie kam zu dem Ergebnis, dass fast 70\% der Ladezeit nur durch warten auf das Netzwerk verbracht wird:

		\begin{figure}[htbp]
			\begin{center}
				\includegraphics[width=0.8\textwidth]{time_on_network.jpg}
				\caption{Untersuchung der top 1 Millionen Alexa Seiten (Abbildung von: \autocite{alexa})}
				\label{fig:time_on_network}
			\end{center}
		\end{figure}

		Es macht also durchaus Sinn, sich diesen Bereich näher anzusehen, um zu verstehen worauf Einfluss genommen werden kann und wo nicht.


		\subsubsection{Mobilfunknetz} % (fold)
		\label{ssub:Mobilfunknetz}
			Es gibt unterschiedliche Mobilfunktsandards mit denen Anwender Anbindung an das Internet erlangen. Aber selbst wenn einem Anwender 4G vom Mobilfunkanbieter versprochen wird, so ist die Netzabdeckung mit 4G noch nicht vollständig deckend. Das bedeutet, dass der Anwender auf ein niedrigeres Netz wie zum Beispiel 3G ausweichen muss. Die verschiedenen Netzwerke unterscheiden sich entscheidend in ihrer Datenrate und vor allem in der Latenz. Die Tabelle in Abbildung \ref{fig:mobile_networks} gibt eine Übersicht:

			\begin{figure}[htbp]
				\begin{center}
					\includegraphics[width=\textwidth]{mobile_networks.jpg}
					\caption{Datenrate und Latenz für eine aktive mobile Verbindung (Eigene Abbildung nach Tabelle \autocite{grigorikGNetwork})}
					\label{fig:mobile_networks}
				\end{center}
			\end{figure}

			Unser Smartphone ist nicht ständig mit dem "`wireless service provider"' verbunden. Ist eine erste Verbindung nötig, so muss das Smartphone dem Sendeturm mitteilen, dass es Kommunizieren möchte. Der Anbieter muss die Anfrage Authentifizieren, die Verbindung herstellen und dann die Anfrage in das Internet weiter leiten. Die Zeit bis eine Authentifizierung erfolgt ist, kann je nach Anbieter und Mobilfunkstandard zwischen <100ms (LTE) und 2,5 Sekunden (3G) liegen \autocite{grigorikRadio}! Bereits hier ist zu sehen, dass es "`worst case"' Szenarien gibt, durch die es nicht möglich sein kann, dass eine Webanwendung in unter einer Sekunde eine Rückmeldung gibt. Gerade Mobilfunknetze unterliegen Stoßzeiten, die Funksignale von Smartphones können sich gegenseitig stören oder das Signal kann in gewissen Gegenden stärker oder schwächer sein.

		% subsubsection Mobilfunknetz (end)


		\subsection{Der HTTP-Request} % (fold)
		\label{sub:der_http_request_komponente}
			Nachdem uns unser Mobilfunkanbieter mit dem Internet verbunden hat, kann die eigentliche Anfrage an den Server gestellt werden.

			\begin{figure}[htbp]
				\begin{center}
					\includegraphics[width=0.8\textwidth]{http_components.jpg}
					\caption{Anfrage der HTML-Datei von Irland mittels 3G Netz (Abbildung nach \url{http://webpagetest.org})}
					\label{fig:http_components}
				\end{center}
			\end{figure}

			\begin{itemize}
				\item DNS Lookup: Um eine Verbindung mit dem Server herzustellen benötigt das HTTP Protokoll die IP Adresse des Ziels. Das heißt der DNS Server wird für den Namen "`http://andreaslorer.de"' die zu diesem Namen zugehörige IP Adresse zurückgegeben.

				\item Initial Connection bezeichnet die Zeit die vergeht, bis eine neue Verbindung zum Server hergestellt wurde damit eine Kommunikation zwischen Browser und Server stattfinden kann. Hierbei findet der sogenannte TCP "`Three-Way-Handshake"' statt, der dafür einen Round Trip benötigt.

				\item TTFB: Ist die Abkürzung für "`Time to first byte"'. Dieser Begriff beschreibt die Zeit die vergeht, bis das erste Byte vom Server beim Browser ankommt. Der Server muss den Request erst zusammenstellen bevor er ihn versenden kann. Dafür werden unter umständen Daten aus der Datenbank abgefragt oder es müssen berechnungen stattfinden. Diese Faktoren beeinflussen die TTFB und kann optimiert werden (schnellerer Server, bessere Datenbankanbindung, Caching).

				\item Content Download: Die Zeit die benötigt wird bis die Datei vom Server Heruntergeladen wurde.

				\item Nachdem das HTML Dokument heruntergeladen wurde, muss es vom Browser noch gelesen und interpretiert werden. Diese Zeit taucht im Diagramm nicht auf.
			\end{itemize}
					
		% subsection http_request_komponente (end)
	% section netzwerke (end)

	\subsection{Das Herunterladen einer 40 Kilobyte Datei} % (fold)
	\label{sub:das_herunterladen_einer_40_kb_datei}

		Abbildung ~\ref{fig:fetching_40kb} zeigt Schematisch wie eine 40kb Datei mittels einer neuen TCP Verbindung heruntergeladen wird. Sie soll verdeutlichen, wie vor allem die RTT eine entscheidende Rolle spielt und warum die Latenz die Geschwindigkeit einer Seite viel höher beeinflusst als die Bandbreite.

		\begin{figure}[htbp]
			\begin{center}
				\includegraphics[width=\textwidth]{fetching_40kb.jpg}
				\caption{Herunterladen von 40kb mittels TCP (Abbildung nach \autocite{grigorikTCP})}
				\label{fig:fetching_40kb}
			\end{center}
		\end{figure}

		Zuerst erfolgt der DNS Lookup, dann muss TCP per \texttt{three way handshake} eine Verbindung aufbauen. Dies kostet bereits 2 round trips, was in diesem Beispiel 400 ms entspricht.
		Durch den TCP slow start (siehe Punkt: \ref{sub:tcp_slow_start}) steht bei einer neuen TCP Verbindung nicht die volle Bandbreite zur Verfügung. Deshalb kann die volle Datenmenge nicht auf einmal, sondern nur durch zusätzliche round trips heruntergeladen werden. Wenn die Performance einer Webanwendung verbessert werden soll, macht es also Sinn in round trips zu denken. Wieviel round trips sind nötig, bis ich dem Browser Informationen übermittelt habe, so dass dieser etwas anzeigen kann? Idealerweise genau einer. 
		
		\begin{figure}[htbp]
			\begin{center}
				\includegraphics[width=\textwidth]{1000ms_budget.jpg}
				\caption{*control plane latency: Wenn noch keinerlei Verbindung zu einem Sendeturm aufgebaut wurde, entstehen einmalige Authentifizierungskosten (Eigene Abbildung nach Daten von: \autocite{venturebeat}\autocite[p. 7, 12]{grigorikRadio})}
				\label{fig:1000ms_budget}
			\end{center}
		\end{figure}

		Wie in Abbildung \ref{fig:1000ms_budget} zu sehen ist bleibt von dem 1000 Millisekunden Budged nicht mehr viel übrig, wenn alleine die Netzwerkzeiten abgezogen werden. Für Nutzer mit 3G Netz ist es laut dieser These selbst im "`best case"' Szenario nicht möglich, die 1000 ms Marke zu durchbrechen. Vor allem wenn man bedenkt, dass das 3G Netz eine Latenz von 200 bis 500 ms haben kann und hier schon der bestmögliche Wert von 200 ms verwendet wurde.\\
		Für Nutzer des 4G schaut es besser aus und es bleiben 445 ms übrig.\\
		Das bedeutet es müssen in den ersten Kilobytes, soviel nützliche Informationen vorhanden sein, damit der Browser bereits anfangen kann mit dem rendering zu beginnen, obwohl noch nicht alle Daten heruntergeladen sind. Noch spitzer formuliert: Der Browser sollte mit den ersten 14 KB (das ist die Menge an Daten die der erste round trip transportieren kann, siehe Abbildung ~\ref{fig:fetching_40kb}) bereits den \texttt{above the fold} bereich rendern können. Um das zu ermöglichen ist es nötig, den Kritischen Rendering-Pfad zu optimieren.
		
	% subsection das_herunterladen_einer_40_kb_datei (end)

	\subsection{Zusammengefasst} % (fold)
	\label{sub:zusammengefasst}
		Dieses Kapitel hat aufgezeigt, dass die Bandbreite eine nur untergeordnete Rolle spielt, wenn von schnellen Webanwendungen die Rede ist. Die Ladezeit wird Dominiert von der Latenz und der damit verbundenen round trip time. Diese entscheidet maßgeblich, wie schnell oder wie langsam der Ladevorgang ist.
		Folgendes ergibt sich auf der Ebene des Netzwerks:
		\begin{itemize}
			\item Die Ladezeit wird für mobile Anwender durch die Latenz bestimmt. 4G kann hier bereits zeiten von <100 ms liefern, was das erreichen der 1000ms Barriere enorm erleichtert.
			\item Bei Anwendern die mittels 3G Netzwerk im Internet surfen besteht wenig Möglichkeiten eine Seite in unter einer Sekunde zu übermitteln, denn die Zeit ab dem Touch Event und im Netzwerk kann bereits schon 900 ms betragen. Diese These deckt sich auch mit den Werten, die im Laufe des Projektes gesammelt wurden. Eine Auswertung davon findet unter Punkt \ref{..} bla statt.
			\item Näheres Platzieren der Bits: Durch die benutzung eines CDN's lassen sich Bits und Bytes näher am Endanwender platzieren was die Netzwerkzeiten verringert.
			\item Sage das Nutzerverhalten voraus: Wenn der Anwender in einer Einkaufs-App 3 Schritte zum vollenden des Kaufvorgangs benötigt, dann lässt sich bei Schritt 1 bereits vorhersagen was er für weitere Ressourcen im nächsten Schritt benötigt. Diese Ressourcen könnten bereits geladen werden. Nachteil: Wenn der Nutzer nie zu Schritt 2 oder 3 kommt, wurde das Internetvolumen des Anwenders (für die der Nutzer eventuell zahlt) umsonst belastet.
			\item Wahl eines guten Hostings: Die RTT als auch die \texttt{Server Response Time} sind je nach Anbieter unterschiedlich. Ein gutes Hosting kann hier unter umständen bereits eine enorme Verbesserung bedeuten. Zur not sollte gewechselt werden!

			\begin{figure}[htbp]
				\begin{center}
					\includegraphics[width=0.9\textwidth]{choose_a_good_host.jpg}
					\caption{Verringerung der response time für http://andreaslorer.de (Abbildung nach pingdom.com)}
					\label{fig:choose_a_good_host}
				\end{center}
			\end{figure}
			
			Wie in der Grafik zu sehen ist, sank die Response Zeit meines Hosting Providers von durchschnittlichen fast 400 Millisekunden auf 183 ms. Dies kann mehrere Gründe haben, so kann sich das Routing zum Server geändert haben, der Server kann ein Update erhalten haben oder die Maschine kann gewechselt worden sein. Was letzten endes dazu geführt hat kann nicht genau gesagt werden.

			\item Senden von weniger Daten: Das schnellste Bit ist das, dass nicht gesendet wird. Das zusammenfügen und verkleinern von Javascript und CSS Dateien verringert die Dateigröße. Zudem lassen sich die Daten per \texttt{GZIP} zwischen Browser und Server komprimieren. Aus Abschnitt \ref{sub:http_1_1} wissen wir bereits, dass die Anzahl von parallelen TCP Verbindungen limitiert ist und Abschnitt \ref{sub:tcp_slow_start} hat den Einfluss des TCP slow start gezeigt. Deshalb macht es Sinn, möglichst wenige Dateien auszuliefern, denn jede Datei Benötigt einen extra TCP Verbindungsaufbau und jede neue TCP Verbindung unterliegt dem TCP slow start. Wie dies in der Praxis umgesetzt wird soll unter Punkt \ref{..} konkretisiert werden.

			\item Stelle nützliche bytes zu Verfügung: Wie in Abbildung \ref{fig:fetching_40kb} zu sehen ist, erfolgt mit dem ersten round trip eine Datenübertragung von ca. 14 KB. Optimal ist es, wenn bereits mit dieser ersten Antwort genügend Informationen vorliegen um etwas auf den Bildschirm des Anwenders zu rendern. Das setzt vorraus, dass die HTML Datei nicht größer ist als 14 KB (nach Kompression).

			\item Vermeiden von Weiterleitungen: Abbildung ~\ref{fig:eliminate_redirects} zeigt den Seitenaufruf von hasbro.com. Wie zu sehen ist, gibt es einen HTTP 301 (Wert in Klammer) Response zurück: 

			\begin{quote}
				301 - Moved Permanently: \textit{"`Die angeforderte Ressource steht ab sofort unter der im "`Location"'-Header-Feld angegebenen Adresse bereit (auch Redirect genannt). Die alte Adresse ist nicht länger gültig."'} \autocite{wikipediaHTTP}
			\end{quote}

			\begin{figure}[htbp]
				\begin{center}
					\includegraphics[width=0.6\textwidth]{eliminate_redirects.jpg}
					\caption{Testlauf von hasbro.com: "`Dulles, VA USA - Thinkpad T430 - Chrome - Cable"' via webpagtest.org)}
					\label{fig:eliminate_redirects}
				\end{center}
			\end{figure}
			
			Wenn ein Anwender \url{hasbro.com} eingibt erfolgt der DNS Lookup und die TCP Verbindung wird aufgebaut. Der Browser erfährt dann, dass die Ressource unter einer anderen Adresse zur Verfügung steht. Danach erfolgt die DNS auflösung für www.hasbro.com bei dem der gleiche Vorgang vonstatten geht. Es erfolgt die Weiterleitung auf die jeweilige Ländersprache von www.hasbro.com, wieder mit dem selben Vorgang. Nach rund 900ms konnte die erste Anfrage an die richtige Zieladresse aufgegeben werden! Aber auch Bilder werden auf dieser Seite Weitergeleitet wie in den Anfragen 49, 51 und 53 zu sehen ist. Ruft man sich die RTT von einem 3G Netz ins Gedächtnis dürfte klar werden, was Weiterleitungen für den Smartphonenutzer bedeuten können. \\

			\begin{figure}[htbp]
				\begin{center}
					\includegraphics[width=\textwidth]{eliminate_redirects_smartphone.jpg}
					\caption{Testlauf mittels Smartphone von hasbro.com: "`Dulles VA USA - Modell: MOTOG. 3G shaped 1.6Mbps / 300ms RTT"' Detailierter Test unter: \url{http://www.webpagetest.org/result/150310_AH_HVD/1/details/}}
					\label{fig:eliminate_redirects_smartphone}
				\end{center}
			\end{figure}

			Abbildung \ref{fig:eliminate_redirects_smartphone} zeigt den Aufruf von hasbro.com mittels Smartphone mit 3G Netz. Katastrophale 5.5 Sekunden dauert alleine der Verbindungsaufbau zum HTML Dokument der Seite!\\

			Die Weiterleitung von hasbro.com auf www.hasbro.com ist für die Suchmaschinenoptimierung \footnote{engl. SEO - search engine optimization} allerdings Sinnvoll, denn sonst würde unter zwei Namen der gleiche Inhalt zu finden sein. Dies ist aus sicht von Google "`Duplicated Content"' und kann zu einer Abstrafung im Ranking führen.\footnote{Mehr zu diesem Thema gibt es unter \url{http://www.sem-deutschland.de/seo-tipps/duplicate-content-definition/}}

		\end{itemize}

	% subsection zusammengefasst (end)


	\subsection{Kritischer Rendering-Pfad} % (fold)
	\label{sub:critical_render_path}
		Auf Englisch "`critical render path"' genannt, ist der wohl wichtigste Begriff, wenn es um schnelle Ladezeiten geht. Durch die Optimierung des Rendering-Pfads kann die benötigte Zeit für das erste Rendern der Seite erheblich verkürzt werden. Das Verständnis des Rendering-Pfads ist zudem eine wesentliche Voraussetzung für die Erstellung von schnellen Webanwendungen und soll in diesem Abschnitt ausführlich erklärt werden. Dabei wird der Begriff in seine Teile Zerlegt: Kritischer, Rendering und Rendering-Pfad.\\

		\subsubsection{Rendering} % (fold)
		\label{ssub:rendering}
			Rendering: Der Browser liest das HTML Dokument und übersetzt es. Diesen Vorgang nennt man auch Parsen. "`Das Parsen der HTML- und CSS-Ressourcen und Ausführen von JavaScript beansprucht Zeit und Clientressourcen. Je nach Geschwindigkeit des Mobilgeräts und Komplexität der Seite kann dieser Prozess Hunderte von Millisekunden in Anspruch nehmen."'\autocite{google14} Bei der Optimierung von Webanwendungen ist besonderst das Auftreten des ersten Zeichnens interessant (engl. "`First Paint Event"'). Je früher das sogenannte \texttt{First Paint Event} auftritt umso höher ist die \texttt{Perceived Performance} der Webanwendung.

			\begin{figure}[htbp]
				\begin{center}
					\includegraphics[width=0.75\textwidth]{browser_render.jpg}
					\caption{Der Render Prozess (Eigene Abbildung)}
					\label{fig:browser_render}
				\end{center}
			\end{figure}

		% subsubsection rendering (end)

		\subsubsection{Rendering-Pfad} % (fold)
		\label{ssub:rendering_pfad}

			Der \texttt{Rendering-Pfad} setzt sich aus den für die Anwendung nötigen Ressourcen zusammen. Webanwendungen bestehen schließlich nicht nur aus einer HTML Datei, sondern aus mehreren Javascript und CSS Dateien.

			\begin{figure}[htbp]
				\begin{center}
					\includegraphics[width=\textwidth]{critical_render_path.jpg}
					\caption{Ressourcen die für das Rendern von nöten sind (Eigene Abbildung)}
					\label{fig:critical_render_path}
				\end{center}
			\end{figure}

			Gegeben ist das folgende Beispiel einer simplen HTML Datei.

			\begin{lstlisting}[captionpos=b, caption=Beispiel Code, label=beispielCode]
				<!DOCTYPE html>
				<meta charset="utf-8">
				<title>Web Performance fuer den mobilen Endanwender</title>

				<link href="assets/styles.css" rel="stylesheet" />
				<script src="assets/script.js"></script>

				<p> Hello world! </p>

			\end{lstlisting}

			Nachdem diese Datei heruntergeladen wurde, beginnt der Browser sie von oben nach unten zu Parsen. Dabei stößt er in Zeile 5 auf einen \texttt{Link-Tag} der ihn anweißt diese Datei herunterzuladen. In Zeile 6 findet der Browser einen \texttt{Script-Tag}. Auch diese Datei muss herunterladen, interpretiert und ausgeführt werden, denn jede Javascript Datei kann den DOM-Baum\footnote{Der DOM-Baum: \url{http://wiki.selfhtml.org/wiki/JavaScript/Objekte/DOM}} oder das CSS manipulieren. So können per Javascript sowohl Elemente dem DOM-Baum hinzugefügt, als auch weggenommen werden oder Elemente können eine Änderung ihrer CSS Attribute erhalten. Dieser Umstand verbietet es dem Browser mit dem Rendering zu beginnen, da bis zur Ausführung der Javascript Dateien noch Manipulationen erfolgen können.
			Bevor also das "`Hello world"' in Zeile 8 angezeigt werden kann, ist das rendern blockiert. Dieses Verhalten nennt sich auch "`Render Blocking"' und wird sowohl von Javascript als auch von CSS Dateien ausgelöst. Folglich spricht man hierbei auch von "`Render Blocking Javascript"' und "`Render Blocking CSS"'. Erst wenn diese Blockierenden Ressourcen geladen und interpretiert wurden, kann der Browser mit dem Rendern beginnen. 

		% subsubsection rendering_pfad (end)

		\subsubsection{Critical render path} % (fold)
		\label{ssub:critical_render_path}
			
			\texttt{Critical render path} sind genau die Javascript und CSS Dateien, die für den für das Rendern des \texttt{above the fold} (Punkt: \ref{sub:above_the_fold}) von nöten sind. Um dies umzusetzen ist es nötig, die Ressourcen in zwei Teile zu zerlegen: Für das Rendering \textbf{absolut} notwendig und nicht notwendig. Alle Dateien die nicht notwendig für das erste Rendern sind sollten so lange mit dem Laden verzögert werden, bis die Anwendung geladen ist. Wie genau so eine Umsetzung aussieht, wird in Punkt: \ref{..} ausführlich gezeigt.
		% subsubsection critical_render_path (end)


		\subsubsection{Zusammengefasst} % (fold)
		\label{ssub:zusammengefasst}
		  Folgende Pattern lassen sich, bedingt durch den \texttt{Kritischen Rendering-Pfad}, für die Erstellung von Webanwendungen ableiten.

			\begin{itemize}
				\item CSS Dateien möglichst weit oben im "`<head>"' Bereich platzieren und Javascript vor dem schließen des "`</body> tags"'. Da Javascript und CSS so lange Blockieren, bis sie heruntergeladen wurden, kann mit dem Parsen des gesamten Dokumentes nicht fortgefahren werden. \autocite{placingScripts}\\
				Das Platzieren von Javascript am Ende des Dokuments hat allerdings den Nachteil, dass sich der Zeitpunkt des Herunterladens verzögert. Deshalb ist es ratsam genau die Javascript Dateien in den "`<head>"' zu verlagern, die \texttt{Kritisch} für das Rendern des \texttt{above the fold} Bereichs sind und den rest der Scripte vor den "`</body> tag"'. 

				\item Zusammenfügen von Dateien: Je weniger einzelne Dateien umso weniger wird das Rendern der Seite blockiert.

				\item Aufteilen von Ressourcen in 2 Gruppen: Für das Rendering kritisch und unkritisch. Das gilt sowohl für CSS als auch für Javascript Dateien. Unkritische Dateien werden solange verzögert, bis der above the fold der Seite geladen wurde.

				\item Inlining von CSS im HTML Dokument: Durch das Einbetten von CSS direkt in das HTML Dokument wird das CSS bereits mit der ersten Server Antwort mitgesendet. Dadurch muss der Browser die Datei nicht anfordern und heruntergeladen, sondern kann gleich mit dem Parsen beginnen.

				\item Die Herausforderung besteht darin, in der eigenen Webanwendung die für das Rendern kritischen Ressourcen zu erkennen und aufzuteilen, ohne die Funktionalität der Anwendung in Mitleidenschaft zu ziehen. Dinge die fast immer verzögert geladen werden können sind zum Beispiel Social Media Buttons (Facebook, Google+, Twitter ect.), Widgets oder Tracking Codes wie Google Analytics.
			\end{itemize}

		% subsubsection zusammengefasst (end)

	% subsection critical_render_path (end)

	\subsection{Analyse des Wasserfalls}
	\label{sub:analyse_des_wasserfalls}
		Für ein besseres Verständnis des Kritischen Rendering-Pfads soll ein praktisches Beispiel einer nicht optimierten Seite helfen. In Abbildung ~\ref{fig:wasserfall_old} ist ein Ausschnitt eines Wasserfallmodells dargestellt. 

		\begin{figure}[htbp]
			\begin{center}
				\includegraphics[width=\textwidth]{wasserfall_old.jpg}
				\caption{Testlauf von: "`Dulles VA USA - Modell: MOTOG. 3G shaped 1.6Mbps / 300ms RTT"' Ganzer Test: \url{http://www.webpagetest.org/result/150308_A1_2W4/8/details/}}
				\label{fig:wasserfall_old}
			\end{center}
		\end{figure}
		
	  Die Abbildung zeigt, dass typische Verhalten des Browsers: Es werden zuerst CSS, Javascript und anschließend die Bilder heruntergeladen. Hierbei fällt auf, dass er nicht mit allen Dateien gleichzeitig beginnen kann, sondern wie in Punkt \ref{sub:http_1_1}: HTTP/1.1 nur 6 TCP Verbindungen pro Host Name aufbauen darf. \footnote{Es sind deshalb 7 Verbindungen, da die Datei: "`code.jquery"' von einer Google Domäne kommt und deshalb als neuer Host Name zählt.} Das bedeutet, das alle weiteren Ressourcen auf eine frei werdende TCP Verbindung warten müssen. Je weniger einzelne Dateien die Webseite benötigt, umso weniger bilden sich Warteschlangen für eine frei werdende TCP Verbindung (der Wasserfall wird flacher).\\
	  Auch in diesem Diagramm ist die Latenz als dominierender Faktor zu sehen. Es fällt auf, dass es nur einen relativ langen blauen Balken gibt (Anfrage \#12 gallery.php) bei dem für längere Zeit etwas heruntergeladen wird. Das herunterladen der meisten Inhalte dauert überwiegend nur so lange, wie die Zeit die nötig war um eine Verbindung herzustellen. \\
	  Die senkrechte blaue Line bedeutet: \texttt{Document Complete} und das heißt für den Browser, dass alle für das Rendern nötigen Ressourcen fertig heruntergeladen wurden und nun vorhanden sind. Dadurch kann er bei Sekunde 7 (senkrechte grüne Line) mit dem Rendern beginnen. Das Ziel des Kritischen Rendering-Pfads ist es, diese senkrechte grüne Linie möglichst weit nach links zu schieben, also die Zeit bis zum ersten Rendern zu minimieren.
	  Zum Vergleich nun das Wasserfallmodell einer Optimierten Seite:

	  \begin{figure}[htbp]
	  	\begin{center}
	  		\includegraphics[width=\textwidth]{wasserfall_neu.jpg}
	  		\caption{Selbe Testbedingungen wie bei Abbildung \ref{fig:wasserfall_old}. Ganzer Test: \url{http://www.webpagetest.org/result/150308_5V_JSD/6/details/}}
	  		\label{fig:wasserfall_neu}
	  	\end{center}
	  \end{figure}
	  Wie zu sehen ist, fällt die senkrechte grüne Linie bereits viel früher bei rund 1.6 Sekunden. Das CSS und Javascript wird zu diesem Zeitpunkt noch heruntergeladen. Daraus lässt sich schlussfolgern, dass in dieser optimierten Version kein \texttt{render blocking Javascript / CSS} mehr vorhanden ist. Dadurch ist der Browser nicht blockiert und kann bereits früh mit dem Rendern der Seite beginnen. Dieses Diagramm lässt sich noch viel weiter interpretieren und belegt die Hauptaussage dieses Kapitels "`Brechen der 1000 ms Barriere"':\\
	  Request Nummer 1 zeigt genau dass, was im Kapitel "`\ref{sub:netzwerke} Netzwerke"' beschrieben wurde. 
	  
	  \begin{figure}[htbp]
	  	\begin{center}
	  		\includegraphics[width=0.8\textwidth]{wasserfall_detail.jpg}
	  		\caption{Request Nr. 1 im Detail (Abbildung nach webpagetest.org)}
	  		\label{fig:wasserfall_detail}
	  	\end{center}
	  \end{figure} 

	  Mit gerundeten Werten ergibt sich:

	  \begin{itemize}
	  	\item DNS Lookup: 1 RTT = 300 ms
	  	\item Initial Connection (TCP - 3 Way Handshake): 1 RTT = 300 ms
	  	\item TTFB: Server Processing Time\footnote{die Antwort muss erst vom Server generiert werden}: = 300 ms
	  	\item Content Download\footnote{Das HTML Dokument beträgt 6.3 KB (siehe in Abbildung \ref{fig:wasserfall_detail} Eintrag: Bytes In (downloaded)), TCP kann mit dem ersten round trip rund 14 KB transportieren. Das HTML Dokument kann also in einem round trip geliefert werden.}: 1 RTT = 300 ms 
	  	\item Zeit fürs Parsen, Ausführen und Rendern: ca. 500 ms \footnote{Dies ist in der Abbildung mittels rotem Pfeil (<- ->) makiert}:  
	  \end{itemize}
 
	  Obwohl hier nur eine 6.3kb Datei heruntergeladen wurde und keine 40kb so wie in Beispiel \ref{fig:1000ms_budget}, ist es ohne eine geringere Latenz (4G) nicht möglich unter die 1000 ms Barriere zu kommen. Leider stellt der Serviceanbieter von \url{webpagtest.org} keine Testumgebunt mit 4G Netz zu Verfügung. Ein Beweis für das erreichen eines "`first render"' mittels Smartphone in unter 1000 ms Sekunde steht folglich aus. Mittels Kabelverbindung sind werte um die 300 ms zu erreichen, dies wird per Datenauswertung in Kapitel \ref{..} ? gezeigt.

	  % evt hinzufügen von meinen Daten als Diagramm

	% subsection analyse_des_wasserfalls (end)

% section die_1000_ms_barriere (end)

\pagebreak

    \section{Entwicklung} % (fold)
\label{sec:entwicklung}
	Dieses Kapitel soll den Entwicklungsprozess konkretisieren und den Optimierungsprozess einer Webanwendung aufzeigen. Es soll erläutern, welche Fragen sich stellten und welche Antworten darauf gefunden wurden. Wie bestimmte Probleme gelöst wurden. Welche Tools und Hilfsmittel zur verwendung kamen. Dies soll ein Bewusstsein dafür schaffen, was möglich ist und wie eine technische Umsetzung aussehen kann. 
	
	\subsection{Tools}
	\label{sub:tools}
		Dies ist eine Auflistung an Tools und nützlichen Seiten, die entweder im Projekt verwendung fanden, oder die für Wertvoll befunden wurden und deshalb hier ihren Platz finden, damit jeder für sich entscheiden kann, ob der Einsatz davon sinnvoll sein könnte.

		\subsubsection{Google Chrome Developer Tool} % (fold)
		\label{ssub:google_chrome_developertool}
			Dieses Tool ist über die Taste F12 im Chrome Browser zu finden. Nützliche Features sind: 

			\begin{itemize}
				\item \texttt{Device Emulation} \footnote{Bei geöffnetem Tool (F12): strg + shift + M oder klick auf das Smartphone Symbol}: Damit lassen sich verschiedene Devices wie Smartphones, Ipad oder verschiedene Desktopauflösungen simulieren. Auch das Touch verhalten wird Simuliert.
				\item In der Device Emulation lässt sich auch die Netzwerkgeschwindigkeit simulieren. Dies ist allerdings nur eine Simulation und kann unter wahren Bedingungen stark abweichen.
				\item Netzwerk: Hier lässt sich das Wasserfallmodell nachvollziehen. Auch lässt sich hier das Caching des Browsers abschalten, wärend das Developer Tool geöffnet ist.
				\item Audits: Unter diesem Reiter bekommt man erste Informationen, welche Verbesserungen es für diese Seite aus dem Gesichtspunkt der Performance ergeben. So wird zum Beispiel aufgezeigt, wie viele CSS Selektoren auf dieser Seite gar keine verwendung finden (gerade bei CSS-Frameworks wie Bootstrap kann es sein, dass rund 90\% der Selektoren keine Verwendung haben)
			\end{itemize}
		% subsubsection google_chrome_developertool (end)

		\subsubsection{Google Pagespeed Insight} % (fold)
		\label{ssub:google_pagespeed_insight}
			Pagespeed Insight ist ein Analysetool für Webanwendungen. Per URL Eingabe wird die Anwendung Aufgerufen und gegen die "`Best Practices"' von Google getestet: \footnote{ \url{http://tinyurl.com/nvxksks} }. Dabei wird ein Rating von 1 (schlecht) bis 100 (gut) vergeben. Mobile und Desktop Version werden voneinander unabhängig bewertet. Findet das Tool verstöße gegen die \texttt{best practices}, so gibt es Hilfestellungen wie zum Beispiel weiterführende Links oder Hinweise zur Behebung des Problems. Für die Verbesserung der Perfomance ist dieses Tool eines der besten Anlaufziele, um einen Überblick zu bekommen wo sich die Probleme befinden. Pagespeed Insight gibt es auch als Plugin für das Google Chrome Developer Tool.\footnote{Plugin - Pagespeed Insight: \url{http://tinyurl.com/mv8fcx8}}
		
		% subsubsection google_pagespeed_insight (end)

		\subsubsection{Google Closure Compiler} % (fold)
		\label{ssub:closure_compiler}
			Ein simples Tool von Google\footnote{\url{http://closure-compiler.appspot.com/}}, mit der Aufgabe Javascript zu verkleinern. Dieser Vorgang nennt sich auch "`minify"' und ist auch für HTML und CSS möglich. Ein Beispiel:

			\begin{lstlisting}[captionpos=t, caption=Input, label=lst:minifyInput]
			/**
			 * urlEncodes an object to send it via post
			 * @param  {Object} object Object with key value pairs
			 * @return {String}        string in format key=value&foo=bar
			 */
			var urlEncode = function (object) {
			  var encodedString = '';
			  for (var prop in object) {
			    if (object.hasOwnProperty(prop)) {
			      if (encodedString.length > 0) {
			          encodedString += '&';
			      }
			      encodedString += encodeURI(prop + '=' + object[prop]);
			    }
			  }
			  return encodedString;
			};
			\end{lstlisting}

			Wird zu:

			\begin{lstlisting}[captionpos=t, caption=Output, label=lst:minifyOutput, breaklines=false]
			var urlEncode=function(c){var a="",b;for(b in c)c.hasOwnProperty(b)&&
			(0<a.length&&(a+="&"),a+=encodeURI(b+"="+c[b]));return a};
			\end{lstlisting}


		
		% subsubsection closure_compiler (end)

		\subsubsection{Webpagetest} % (fold)
		\label{ssub:webpagetest}
			Was ist Webpagetest
			Speedindex
			Features
			  Filmstrip
			  Video
			  Tests vergleichen
			  Test History
			  Forum
		% subsubsection webpagetest (end)	

		\subsubsection{Pingdom} % (fold)
		\label{ssub:pingdom}
			Tests wie Webpagtest nur ungenauer :D
			Monitoring feature

		
		% subsubsection pingdom (end)

		\subsubsection{Speedcurve} % (fold)
		\label{ssub:speedcurve}
			frontend monitoring
			Vergleicht zur Konkurrenz
		% subsubsection speedcurve (end)

		\subsubsection{Google Spreadsheet} % (fold)
		\label{ssub:google_spreadsheet}
			Script Schnittstelle
			Bulk Webpagetest
			Link zu Daten geben
		
		% subsubsection google_spreadsheet (end)

		\subsubsection{Feed the Bot} % (fold)
		\label{ssub:feed_the_bot}
			Best practices
				
		% subsubsection feed_the_bot (end)

		\subsubsection{Critical Path CSS Generator} % (fold)
		\label{ssub:critical_path_css_generator}
		
		% subsubsection critical_path_css_generator (end)

		\subsubsection{moz Jpeg} % (fold)
		\label{ssub:moz_jpeg}
		
		% subsubsection moz_jpeg (end)

		\subsubsection{Http Archive \& bigqueri.es} % (fold)
		\label{ssub:http_archive_bigqueri_es}
			
		% subsubsection http_archive_bigqueri_es (end)

		\subsubsection{Perf Tooling Today} % (fold)
		\label{ssub:perf_tooling_today}
		
		% subsubsection perf_tooling_today (end)

		\subsubsection{Twitter} % (fold)
		\label{ssub:twitter}
		
		% subsubsection twitter (end)



	% subsection tools (end)


	\subsection{Ausgangspunkt}
	\label{sub:ausgangspunkt}
	
	% subsection ausgangspunkt (end)

	\subsection{Prozess der Validierung}
	\label{sub:prozess_der_validierung}
	
	% subsection prozess_der_validierung (end)

	

% section entwicklung (end)
    \section{Best-Practices} % (fold)
\label{sec:best_practices}

	\subsection{Serverseitig}
	\label{sub:serverseitig}

		\subsubsection{Verringern von DNS Lookups} % (fold)
		\label{ssub:verringern_von_dns_lookups}
		
		% subsubsection verringern_von_dns_lookups (end)

		\subsubsection{HTTP Requests} % (fold)
		\label{ssub:http_requests}
		pro / contra von sprites und großen concatenated files
		% subsubsection http_requests (end)

		\subsubsection{Caching} % (fold)
		\label{ssub:caching}
		pro / contra / problem bei updates
		% subsubsection caching (end)

		\subsubsection{Pagespeed Mod} % (fold)
		\label{ssub:pagespeed_mod}
		
		% subsubsection pagespeed_mod (end)

		\subsubsection{Images} % (fold)
		\label{ssub:images}

		% subsubsection images (end)


%--------------------------------------------------------
%  subsection serverseitig (end)
%--------------------------------------------------------


	\subsection{Clientseitig}
	\label{sub:clientseitig}
	
	% subsection clientseitig (end)

	\subsection{Unternehmenskultur} % (fold)
	\label{sub:unternehmenskultur}
	When you want to be fast, you have to give up the things slowing you down (addy osmani 14, slide 62)
	% subsection unternehmenskultur (end)

% section best_practices (end)
\pagebreak
    \section{Workflow} % (fold)
\label{sec:workflow}

	\subsection{Performanten Code schreiben}
	\label{sub:performanten_code_schreiben}
	
	% subsection performanten_code_schreiben (end)

	\subsection{Minify und Uglify}
	\label{sub:minify_und_uglify}
	
	% subsection minify_und_uglify (end)

	\subsection{Concatenating}
	\label{sub:concatenating}
	
	% subsection concatenating (end)

	\subsection{Task Manager}
	\label{sub:task_manager}
	
	% subsection task_manager (end)

	\subsection{Dependency Manager}
	\label{sub:dependency_manager}
	
	% subsection dependency_manager (end)

	\subsection{Generators}
	\label{sub:generators}
	
	% subsection generators (end)


% section workflow (end)
	\pagebreak

	\printbibliography

 % 	\bibliographystyle{apalike}
 % 	\bibliography{sections/zitate}
	\end{document}